\begin{itemize}
    \item Die Ansicht des Werkverzeichnisses ist offen und kann für Gäste der Seite auch ohne Log-In stattfinden. 
          Die Komponente \textit{Werkverzeichnisansicht} ermöglicht es, Motivgruppen und Motive zu wählen 
          sowie zwischen den zugehörigen Werken zu wechseln. Dazu kommen die Filter-, Sortier- und Suchfunktionen 
          des Werkverzeichnisses. Neben der standardmäßigen Galerieansicht kann auch zu einer Listenansicht 
          gewechselt werden.
    \item Es können auch PDF-Steckbriefe zu den Werken exportiert werden. Diese Funktionalitäten stecken in 
          der Komponente \textit{PDF-Export}.
    \item Für die Anmeldung als Nutzer sowie als Admin gibt es eine \textit{Authentifizierungskomponente}, 
          die Zugriff auf weitere Komponenten gewährt. Sie übernimmt auch die Registrierung von neuen Nutzern 
          und enthält von Spring bereitgestellte Funktionalitäten für die Nutzerverwaltung und Sicherheitskonfiguration.
    \item Ist der Nutzer angemeldet, hat er Zugriff auf die \textit{Benutzerbereich-Verwaltungskomponente}. 
          Diese ermöglicht es ihm, seine Suchanfragen sowie seine Notizen zu den Werken zu speichern und zu verwalten.
    \item Die \textit{Kunstwerk-Verwaltungskomponente} fasst Funktionalitäten zusammen, die sowohl dem Nutzer 
          als auch dem Admin zur Verfügung stehen. Der Nutzer kann Anfragen zum Hinzufügen, Editieren sowie 
          Löschen eigener Werke stellen. Der Admin kann den Anfragen entweder zustimmen oder diese ablehnen. 
          Außerdem verwaltet der Admin hier, von welchen Werken die Standorte angezeigt werden dürfen.
    \item Auf die \textit{Benutzer-Verwaltungskomponente} hat nur der Admin Zugriff. Diese Komponente erlaubt 
          es ihm, Nutzer zu löschen.
    \item In der Datenbank werden alle Produktdaten gespeichert. Zugriffe laufen über \textit{Repository-Klassen}, 
          die das von Spring zur Verfügung gestellte Repository-Interface erweitern.
    \item Die \textit{Lido-Parser-Komponente} fasst die Funktionalitäten zusammen, die die Werkdaten der 
          digicult-Datenbank abfragen und so aufbereiten, dass sie in unsere Datenbank überführt werden können.
\end{itemize}
\begin{figure}[h]
    \centering
\includegraphics[width=1\textwidth,height=1.5\textheight,keepaspectratio]{img/Komponentendiagramm.png}
    \caption{Komponentendiagramm - A}
    \label{fig:komponentendiagramm-a}
\end{figure}

