\section{Entwicklungsumgebung}\label{sec:entwicklungsumgebung}
\subsection{Einleitung}

Nachfolgend finden Sie die vorläufige Dokumentation für die Entwicklung eines Werkverzeichnisses für den Künstler Droege. Ziel dieses Systems ist es, die Anzeige und Filterung der Kunstwerke zu ermöglichen, Informationen über Droege bereitzustellen und eine Benutzerverwaltung zu ermöglichen.

\subsubsection{Technologien}
In der nachfolgenden Tabelle sind die Technologien, die zur Durchführung des Projekts eingesetzt werden, kurz dargestellt und beschrieben.
\centering
\begin{tabular}{|c|c|}
\hline
\textbf{Technologie} & \textbf{Beschreibung} \\
\hline
Git & Versionskontrollsystem zur kollaborativen Entwicklung \\
Java & Hauptprogrammiersprache für die Backend-Logik \\
Spring Boot & Framework zur Erstellung von Java-Webanwendungen \\
Thymeleaf & Template-Engine für die serverseitige Generierung von HTML \\
JAXB & Dependency/Library für XML Parsing \\
Haskell & Für die Automatisierung von Git-Skripte \\
Bash & Skripterstellung für Automatisierung und Deployment \\
Gradle & Build-Tool zur Abhängigkeitsverwaltung und Projektkompilierung \\
Docker & Deployment-Tool \\
HTML, CSS & Gestaltung und Strukturierung der Benutzeroberfläche \\
Jakarta Persistence & Für die Objektpersistenz\\
\hline
\end{tabular}

\vspace{1em}

\centering
\begin{tabular}{|c|c|c|}
\hline
\textbf{Technologie} & \textbf{Version} & \textbf{URL} \\
\hline
Git & client dependant & \href{https://git-scm.com/}{git} \\
Java & jdk21 & \href{https://www.oracle.com/de/java/technologies/downloads/}{java} \\
Spring Boot & 3.4.3 & \href{https://start.spring.io/}{Spring Boot} \\
Thymeleaf & 3.1.3 & \href{https://www.thymeleaf.org/}{Thymeleaf} \\
JAXB & v2 & \href{https://github.com/javaee/jaxb-v2}{JAXB} \\
Haskell & The Glorious Glasgow Haskell Compilation System, version 9.4.8 & \href{https://downloads.haskell.org/ghc/latest/docs/users_guide/ghci.html}{Haskell/ghci} \\
Bash & client-dependent & OS-vorinstalliert \\
Gradle & 8.12.1 & \href{https://gradle.org/}{gradle} \\
Docker & 3.8 &  \href{https://www.docker.com/}{Docker} \\
HTML, CSS & HTML5 \& CSS3 & Browser spezifisch \\
Jakarta Persistence & 3.1.0 & \href{https://jakarta.ee/}{Jakarta}\\
\hline
\end{tabular}

\raggedright

\subsection{UML-Diagramme}
Die Entwurfsdokumentation enthält verschiedene UML-Diagramme, die die Systemarchitektur darstellen, darunter:

\begin{itemize}
\item \textbf{Klassendiagramme}: Zeigen die Struktur der Klassen und deren Beziehungen.
\item \textbf{Komponentendiagramme}: Veranschaulichen die logische Organisation der Softwaremodule.
\item \textbf{Verteilungsdiagramme}: Beschreiben die physische Verteilung der Softwarekomponenten auf verschiedenen Systemen.
\item \textbf{Sequenzdiagramme}: Stellen den Ablauf der Interaktionen zwischen den Systemkomponenten dar.
\end{itemize}

Diese Diagramme dienen dazu, die Architektur, Abhängigkeiten und Interaktionsabläufe des Systems verständlich zu machen.

