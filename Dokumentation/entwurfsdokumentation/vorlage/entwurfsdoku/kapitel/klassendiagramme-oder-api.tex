\begin{figure}[h]
    \centering
\includegraphics[width=1\textwidth,height=1.5\textheight,keepaspectratio]{img/Klassendiagramm.jpg}
    \caption{Klassendiagramm}
    \label{fig:Klassendiagramm}
\end{figure}

\begin{table}[h]
	\centering
	\begin{tabularx}{\textwidth}{X X}
		\rowcolor[HTML]{C0C0C0} 
		\textbf{Name} & \textbf{Aufgabe} \\
		(Package) Model & Enthält Klassen mit Grundstruktur verwendeter Daten. \\
		\rowcolor[HTML]{E7E7E7} 
		(Package) Repository & Stellt verschiedene Datenbanken zum Speichern und Abrufen einzelner Daten zur Verfügung. \\
		LIDOParser & Ein Service, welcher gespeicherte APIs durchgeht, Lido-Dateien abruft, in Werk-Objekte überführt und persistent speichert. \\
		\rowcolor[HTML]{E7E7E7} 
		PDF-Export & Ein Service, welcher zu einem übergebenen Werk eine PDF-Datei erstellt \\
		HomeController & Übernimmt die Aufgabe, die Interaktionen des Anwenders mit der Hauptseite zu verarbeiten. Z.B. das Wählen einer Motivgruppe. \\
		\rowcolor[HTML]{E7E7E7} 
		MotivgruppenController & Übernimmt die Aufgabe, die Interaktionen des Anwenders mit der Motivgruppenseite zu verarbeiten. Z.B. das Wählen einer Referenzbildes. \\
		MotivController & Übernimmt die Aufgabe, die Interaktionen des Anwenders mit der Werkseite zu verarbeiten. Z.B. das Erstellen einer PDF-Datei zu einem Werk. \\
        \rowcolor[HTML]{E7E7E7} 
        AdminController & Übernimmt die Aufgabe, die Interaktionen des Admins mit der Adminseite zu verarbeiten. Z.B. Interaktionen mit Anwenderdatenbank oder aktualisieren der Werkdatenbank. \\
        UserController & Übernimmt die Aufgabe, die Interaktionen des Nutzers mit der Nutzerseite zu verarbeiten. Z.B. Senden von Anfragen. \\
        \rowcolor[HTML]{E7E7E7} 
        AuthController & Übernimmt Anfragen zum Registrieren oder Einloggen des Anwenders.
	\end{tabularx}
	\caption{Klassenbeschreibung}
	\label{table:klassenbeschreibung}
\end{table}

